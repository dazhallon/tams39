\message{ !name(examination.tex)}\documentclass[two column]{report}
\usepackage[utf8]{inputenc}

\usepackage{amsmath,amsthm,amssymb}

\usepackage{subcaption}
\usepackage{graphicx}

\usepackage{diagbox}

\usepackage{biblatex}

\bibliography{ref}

\graphicspath{{./images/}}

\usepackage{listings}


\usepackage{color}

\definecolor{gray}{rgb}{0.5,0.5,0.5}
\definecolor{orange}{rgb}{0.8,0,0}

\lstdefinestyle{matlab}{
  belowcaptionskip=1\baselineskip,
  breaklines=true,
  frame=L,
  xleftmargin=\parindent,
  language=octave,
  showstringspaces=false,
  basicstyle=\footnotesize\ttfamily,
  keywordstyle=\bfseries\color{green},
  commentstyle=\color{gray},
  identifierstyle=\color{blue},
  stringstyle=\color{orange},
}
\lstdefinestyle{R}{
  belowcaptionskip=1\baselineskip,
  breaklines=true,
  frame=L,
  xleftmargin=\parindent,
  language=R,
  showstringspaces=false,
  basicstyle=\footnotesize\ttfamily,
  keywordstyle=\bfseries\color{green},
  commentstyle=\color{gray},
  identifierstyle=\color{blue},
  stringstyle=\color{orange},
}

\def\listingsfont{\ttfamily} 
\def\listingsfontinline{\ttfamily}

\title{TAMS39 - Examination exercises}
\author{Anton Karlsson\\antka388\\931217-7117}
\date{}
\begin{document}

\message{ !name(ex4.tex) !offset(-33) }
\begin{align*}
{\bf l_1^T x_0 + c_1} = (-0.57, 1.80, 1.01, 6.33, 18.94, 0.33){\bf x}_0 + -703.95 \\ 
{\bf l_2^T x_0 + c_2} = (-0.14, 1.31, 1.48, 5.15, 18.31, 0.04){\bf x}_0 + -554.06 \\ 
{\bf l_3^T x_0 + c_3} = (-1.08, 1.89, 3.03, 5.69, 15.11, 0.51){\bf x}_0 + -618.43, 
\end{align*}
In which we can conclude that ${\bf x_0}$ belongs to  $\pi_i$ if ${\bf
  l_i^T x_0 + c_i} = \max_i \left( {\bf  l_i^T x_0 + c_i}\right)$
\lstinputlisting[style=matlab]{./../ex4.m}
%%% Local Variables:
%%% mode: latex
%%% TeX-master: "examination"
%%% End:

\message{ !name(examination.tex) !offset(-8) }

\end{document}
%%% Local Variables:
%%% mode: latex
%%% TeX-master: t
%%% End:

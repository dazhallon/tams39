
\subsection*{Exercise 8}
\label{sec:exercise-8}

Considering the correlation matrix $\b R$ given, we get that the top
PC's that covers at least 85 \% of the total variance is 
\begin{equation*}
  \begin{pmatrix}
    0.03 &-0.05 &0.04 &-0.09 &0.26 \\ 
    0.26 &0.05 &-0.15 &-0.00 &0.22 \\ 
    0.11 &0.26 &0.40 &-0.02 &0.20 \\ 
    0.20 &-0.04 &-0.08 &-0.08 &0.25 \\ 
    0.08 &0.02 &-0.11 &-0.09 &0.25 \\ 
    0.03 &-0.05 &-0.12 &-0.08 &0.26 \\ 
    0.29 &-0.03 &-0.10 &0.45 &0.18 \\ 
    0.09 &0.32 &-0.08 &-0.28 &0.24 \\ 
    -0.21 &0.40 &-0.37 &0.05 &0.21 \\ 
    -0.01 &-0.06 &-0.11 &-0.20 &0.25 \\ 
    0.10 &-0.11 &-0.20 &-0.01 &0.25 \\ 
    -0.16 &-0.12 &0.21 &-0.16 &0.20 \\ 
    -0.13 &-0.13 &0.23 &-0.11 &0.22 \\ 
    -0.08 &-0.15 &0.23 &-0.05 &0.24 \\ 
    -0.31 &-0.52 &-0.45 &0.25 &0.16 \\ 
    0.12 &-0.26 &0.40 &0.31 &0.18 \\ 
    0.04 &-0.05 &0.18 &0.32 &0.21 \\ 
    -0.43 &0.12 &0.07 &-0.16 &0.20 \\ 
    -0.28 &-0.11 &0.09 &-0.12 &0.24 \\ 
    0.43 &0.12 &-0.09 &0.03 &0.18 \\ 
    -0.35 &0.46 &0.05 &0.55 &0.1
  \end{pmatrix},
\end{equation*}
given in reversed order. So the main PC is in the
right-most column. We interpret these PCs as following
\begin{enumerate}
\item The first PC is very even along the measurements, which makes it
  hard say anything concrete. But we might consider this PC as the
  order to prioritize the measurements, e.g. we might want to measure
  measurements number 1 \textit{Total length, tip of snout to notch of
    flukes} rather then the measurement number 21 \textit{Length of base
    dorsal fin}, since we will get more characteristic that makes sperm
  whales differ from each other.
\item For the second PC (which  covers about 5 \% of the total variance), we notice how messurement 7 and 21 are very
  large compared to the rest. Messurement 7 represents \textit{Center of
    eye to center of ear} and 21 \textit{Length of base  dorsal fin}. It
  is again very hard to say anything meaningful, but we could guess that
  this PC represents measurements of outer regions of the
  whale. Especially the flukes, as measurements number 15-17 is also
  highly represented.
\end{enumerate}
The rest of the main components are very hard to interpret and only
covers less than 5 \% of the total variance.

\subsection*{(b)}
\label{sec:b-7}
 We can construct the confidence interval for the first PC by using
 that
 \begin{equation*}
   \hat\lambda_i \in N(\lambda_i, 2\lambda^2_i/n),
 \end{equation*}
we get
\begin{equation*}
 1- \alpha =  P\left(\frac{|\hat\lambda_i - \lambda_i|}{\lambda_i\sqrt{2/n}} < c\right) =
 P\left(\frac{\hat\lambda_i}{1 + c\sqrt{2/n}} < \lambda_{i} < \frac{\hat\lambda_i}{1 - c\sqrt{2/n}}\right),
\end{equation*}
where $c = \phi^{-1}(1 - \alpha/2)$. The confidence interval for $i =
1$ (the first principal value), becomes
\begin{equation*}
  I_{\lambda_{1}} =
  \begin{pmatrix}
    10.50 &21.26 
  \end{pmatrix}
, \quad \text{width }\alpha = 0.05,\ n = 67.
\end{equation*}

\subsection*{(c)}
\label{sec:c-7}
The asymptotic distribution of $ \hat{\b h_1}$ is given by $\sqrt n
(\hat{\b{h}}_i - \b h_i) \in N_p(\b 0, \b E_i)$
We calculate the matrix $\b E$ as
\begin{equation*}
  \b E = \lambda_{21} \sum_{k \neq 21}\frac{\lambda_k}{(\lambda_k -
    \lambda_{21})^2 }\b h_k \b h_k^T
\end{equation*}
The matrix is to big to been displayed here, so it is recommended to
study it in \texttt{matlab} using the the attached code in
\texttt{ex8.m}.

\subsection*{(d)}
\label{sec:d-3}

The conventional way of calculating the correlation between the PCs and
the original sample data can not be done here since we only have access
to the correlation matrix $\b R$. However, we ca still give indications
of the PCs correlates with the data. From Results 8.3 in Chapter 8 in
\cite[p.433]{book}, we see that correlations are approximately
\begin{comment}
\begin{equation*}
  \begin{pmatrix}
    0.09 &0.02 &-0.04 &0.04 &-0.10 &3.63 \\ 
    0.21 &0.21 &0.04 &-0.15 &-0.00 &3.06 \\ 
    -0.31 &0.09 &0.21 &0.40 &-0.02 &2.79 \\ 
    0.12 &0.16 &-0.04 &-0.08 &-0.09 &3.48 \\ 
    0.10 &0.06 &0.02 &-0.11 &-0.11 &3.47 \\ 
    0.10 &0.03 &-0.04 &-0.12 &-0.09 &3.62 \\ 
    0.16 &0.23 &-0.03 &-0.10 &0.51 &2.56 \\ 
    0.03 &0.07 &0.26 &-0.07 &-0.32 &3.43 \\ 
    -0.05 &-0.16 &0.33 &-0.37 &0.06 &3.01 \\ 
    0.00 &-0.01 &-0.05 &-0.11 &-0.23 &3.50 \\ 
    -0.15 &0.08 &-0.09 &-0.20 &-0.01 &3.54 \\ 
    0.26 &-0.13 &-0.10 &0.21 &-0.19 &2.82 \\ 
    0.06 &-0.10 &-0.10 &0.23 &-0.12 &3.15 \\ 
    0.11 &-0.06 &-0.12 &0.23 &-0.06 &3.33 \\ 
    -0.19 &-0.24 &-0.43 &-0.44 &0.29 &2.30 \\ 
    -0.08 &0.09 &-0.22 &0.40 &0.35 &2.52 \\ 
    -0.05 &0.03 &-0.04 &0.18 &0.37 &2.96 \\ 
    -0.17 &-0.34 &0.10 &0.06 &-0.18 &2.79 \\ 
    -0.15 &-0.22 &-0.09 &0.09 &-0.14 &3.31 \\ 
    -0.37 &0.34 &0.10 &-0.09 &0.04 &2.51 \\ 
    0.14 &-0.28 &0.38 &0.05 &0.64 &1.88  
  \end{pmatrix}.
\end{equation*}
\end{comment}
\begin{equation*}
  \begin{pmatrix}
    0.10 &0.03 &-0.05 &0.04 &-0.09 &0.97 \\ 
    0.25 &0.24 &0.05 &-0.15 &-0.00 &0.82 \\ 
    -0.36 &0.10 &0.24 &0.40 &-0.02 &0.74 \\ 
    0.14 &0.18 &-0.04 &-0.08 &-0.08 &0.93 \\ 
    0.12 &0.07 &0.02 &-0.11 &-0.10 &0.92 \\ 
    0.12 &0.03 &-0.04 &-0.12 &-0.08 &0.97 \\ 
    0.18 &0.26 &-0.03 &-0.10 &0.48 &0.68 \\ 
    0.04 &0.08 &0.29 &-0.08 &-0.30 &0.92 \\ 
    -0.06 &-0.18 &0.36 &-0.37 &0.05 &0.80 \\ 
    0.01 &-0.01 &-0.06 &-0.11 &-0.21 &0.93 \\ 
    -0.17 &0.09 &-0.10 &-0.20 &-0.01 &0.95 \\ 
    0.30 &-0.14 &-0.10 &0.21 &-0.17 &0.75 \\ 
    0.07 &-0.11 &-0.11 &0.23 &-0.11 &0.84 \\ 
    0.12 &-0.07 &-0.13 &0.23 &-0.05 &0.89 \\ 
    -0.22 &-0.27 &-0.47 &-0.45 &0.27 &0.61 \\ 
    -0.09 &0.11 &-0.24 &0.40 &0.33 &0.67 \\ 
    -0.06 &0.03 &-0.04 &0.18 &0.35 &0.79 \\ 
    -0.20 &-0.38 &0.11 &0.07 &-0.17 &0.74 \\ 
    -0.17 &-0.25 &-0.10 &0.09 &-0.13 &0.88 \\ 
    -0.42 &0.38 &0.11 &-0.09 &0.03 &0.67 \\ 
    0.16 &-0.31 &0.42 &0.05 &0.59 &0.50  
  \end{pmatrix}
\end{equation*}
Remember, the right-most column is the correlation between the first PC
and the real data.
%%% Local Variables:
%%% mode: latex
%%% TeX-master: "examination"
%%% End:

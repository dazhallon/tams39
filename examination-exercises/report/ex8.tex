
\subsection*{Exercise 8}
\label{sec:exercise-8}

Considering the correlation matrix $\b R$ given, we get that the top
PC's that covers at least 85 \% of the total variance is 
\begin{equation*}
  \begin{pmatrix}
    0.03 &-0.05 &0.04 &-0.09 &0.26 \\ 
    0.26 &0.05 &-0.15 &-0.00 &0.22 \\ 
    0.11 &0.26 &0.40 &-0.02 &0.20 \\ 
    0.20 &-0.04 &-0.08 &-0.08 &0.25 \\ 
    0.08 &0.02 &-0.11 &-0.09 &0.25 \\ 
    0.03 &-0.05 &-0.12 &-0.08 &0.26 \\ 
    0.29 &-0.03 &-0.10 &0.45 &0.18 \\ 
    0.09 &0.32 &-0.08 &-0.28 &0.24 \\ 
    -0.21 &0.40 &-0.37 &0.05 &0.21 \\ 
    -0.01 &-0.06 &-0.11 &-0.20 &0.25 \\ 
    0.10 &-0.11 &-0.20 &-0.01 &0.25 \\ 
    -0.16 &-0.12 &0.21 &-0.16 &0.20 \\ 
    -0.13 &-0.13 &0.23 &-0.11 &0.22 \\ 
    -0.08 &-0.15 &0.23 &-0.05 &0.24 \\ 
    -0.31 &-0.52 &-0.45 &0.25 &0.16 \\ 
    0.12 &-0.26 &0.40 &0.31 &0.18 \\ 
    0.04 &-0.05 &0.18 &0.32 &0.21 \\ 
    -0.43 &0.12 &0.07 &-0.16 &0.20 \\ 
    -0.28 &-0.11 &0.09 &-0.12 &0.24 \\ 
    0.43 &0.12 &-0.09 &0.03 &0.18 \\ 
    -0.35 &0.46 &0.05 &0.55 &0.1
  \end{pmatrix},
\end{equation*}
given in reversed order\footnote{The order is reversed because $\b R$
  was near-singualr and som computational errer casused \texttt{matlab}
  to compute the som eigenvalues of $\b R$ as negative. \texttt{matlab},
  because of this reason, decided to give the eigenvalues (with
  eigenvectors) from smallest to largest.}. So the main PC is in the
right-most column. We intrepet these PCs as following
\begin{enumerate}
\item The first PC is very even along the measurements, which makes it
  hard say anything concrete. But we might consider this PC as the
  order to prioritice the messurements, e.g. we might want to meassure
  messuremnts number 1 \textit{Total length, tip of snout to notch of
    flukes} rather then the messuremnt number 21 \textit{Length of base
    dorsal fin}, since we will get more carachtersitcs that makes sperm
  wahles differ from each other.
\item For the second PC (which  covers about 5 \% of the total variance), we notice how messurement 7 and 21 are very
  large compared to the rest. Messurement 7 represents \textit{Center of
    eye to center of ear} and 21 \textit{Length of base  dorsal fin}. It
  is again very hard to say anything meaningful, but we could guess that
  this PC represents messuremetns of outer regions of the
  whale. Especially the flukes, as messuremnts number 15-17 is also
  higly represented.
\end{enumerate}
The rest of the main components are very har to intrpret and only
covers less than 5 \% of the total variance.

\subsection*{(b)}
\label{sec:b-7}
 We can construxt the confidence interval for the first PC by using
 that
 \begin{equation*}
   \hat\lambda_i \in N(\lambda_i, 2\lambda^2_i/n),
 \end{equation*}
we get
\begin{equation*}
 1- \alpha =  P\left(\frac{|\hat\lambda_i - \lambda_i|}{\lambda_i\sqrt{2/n}} < c\right) =
 P\left(\frac{\hat\lambda_i}{1 + c\sqrt{2/n}} < \lambda < \frac{\hat\lambda_i}{1 - c\sqrt{2/n}}\right),
\end{equation*}
where $c = \phi^{-1}(1 - \alpha/2)$. The confidence inerval becomes
\begin{equation*}
  I_{\lambda_{i}} = (10.14, 22.95), \quad \alpha = 0.05.
\end{equation*}

\subsection*{(c)}
\label{sec:c-7}
The asymptotic distribution of $ \hat{\b h_1}$ is given by $\sqrt n
(\hat{\b{h}}_i - \b h_i) \in N_p(\b 0, \b E_i)$
We calculate the matrix $\b E$ as
\begin{equation*}
  \b E = \lambda_{21} \sum_{k \neq 21}\frac{\lambda_k}{(\lambda_k -
    \lambda_{21})^2 }\b h_k \b h_k^T
\end{equation*}
The matrix is to big to been displayed here, so it is recomended to
study it in \texttt{matlab} using the the attached code in
\texttt{ex8.m}.

\subsection*{(d)}
\label{sec:d-3}

The conventional way of calculating the correlation between the PCs and
the orginal sample data can not be done here since we only have access
to the correaltion matrix $\b R$. However, we ca still give indications
of the PCs correaltes with the data. From Results 8.3 in Chapter 8 in
\cite[p.433]{book}, we see that correlations are approxemetly
\begin{equation*}
  \begin{pmatrix}
    0.09 &0.02 &-0.04 &0.04 &-0.10 &3.63 \\ 
0.21 &0.21 &0.04 &-0.15 &-0.00 &3.06 \\ 
-0.31 &0.09 &0.21 &0.40 &-0.02 &2.79 \\ 
0.12 &0.16 &-0.04 &-0.08 &-0.09 &3.48 \\ 
0.10 &0.06 &0.02 &-0.11 &-0.11 &3.47 \\ 
0.10 &0.03 &-0.04 &-0.12 &-0.09 &3.62 \\ 
0.16 &0.23 &-0.03 &-0.10 &0.51 &2.56 \\ 
0.03 &0.07 &0.26 &-0.07 &-0.32 &3.43 \\ 
-0.05 &-0.16 &0.33 &-0.37 &0.06 &3.01 \\ 
0.00 &-0.01 &-0.05 &-0.11 &-0.23 &3.50 \\ 
-0.15 &0.08 &-0.09 &-0.20 &-0.01 &3.54 \\ 
0.26 &-0.13 &-0.10 &0.21 &-0.19 &2.82 \\ 
0.06 &-0.10 &-0.10 &0.23 &-0.12 &3.15 \\ 
0.11 &-0.06 &-0.12 &0.23 &-0.06 &3.33 \\ 
-0.19 &-0.24 &-0.43 &-0.44 &0.29 &2.30 \\ 
-0.08 &0.09 &-0.22 &0.40 &0.35 &2.52 \\ 
-0.05 &0.03 &-0.04 &0.18 &0.37 &2.96 \\ 
-0.17 &-0.34 &0.10 &0.06 &-0.18 &2.79 \\ 
-0.15 &-0.22 &-0.09 &0.09 &-0.14 &3.31 \\ 
-0.37 &0.34 &0.10 &-0.09 &0.04 &2.51 \\ 
0.14 &-0.28 &0.38 &0.05 &0.64 &1.88  
  \end{pmatrix}.
\end{equation*}
Remeber, the right-mosrt column is the correlation between the first PC
and the real data.
%%% Local Variables:
%%% mode: latex
%%% TeX-master: "examination"
%%% End:

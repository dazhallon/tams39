\section*{Exercise 4}
\label{sec:exercise-4}

\subsection*{(a)}
\label{sec:a-3}

To check if the covariance matrices can be pooled, the hypothesis
$H_{0}: \Sigma_{1} =\Sigma_{2}  = \Sigma_{3}$ is considered. We
reject $H_{0}$ if
\begin{equation*}
  \Lambda^* = \frac{\prod_{i = 1}^{3} \abs{V_i}^{f_i/2}}{\abs{V}^{f/2}}
  \frac{f^{pf/2}}{\prod_{i = 1}^{3} f_i ^{pf_{i}/2}} < c,\quad
  \text{for some } c,
\end{equation*}
where $f_{i} = n_{i} - 1$, are the degrees of freedom for group $i = 1,
2, 3$, $f = \sum_{i=1}^{3}f_{i}$, and
\begin{equation*}
  \b V_{i} = \b X_{i} (\b I_{n_{i}}  - \frac 1n_{i}\b 1_{n_{i}} \b 1_{n_{i}}^{T})^{T}
  \b X_{i}^{T},\quad i = 1,2,3,
\end{equation*}
and
\begin{equation*}
  \b V = \sum_{i=1}^{3} \b V_{i}.
\end{equation*}
The test $\Lambda^{*} < c$, for some $c$, is an unbiased test.
Further, we can use the correction found in Lecture 5, where
\begin{equation*}
  -2 \frac{m}{f} \ln \Lambda^{*} \sim \chi^{2}(df), \quad df = \frac{1}{2}p(p+1)(r-1),
\end{equation*}
where $m = f- 2\alpha$, where
\begin{equation*}
  \alpha = 
  \left(
    \sum_{i = 1}^{3} \frac{f}{f_{i}} - 1
  \right)
  \frac{2p^{2} + p -1}{12(p+1)(3-1)}.
\end{equation*}
Then 
\begin{equation*}
   -2 \frac{m}{f} \ln \Lambda^{*} = 49.26
\end{equation*}
and
\begin{equation*}
  \chi_{df}^{2}(1-\alpha) = 91.670, \quad \alpha = 0.05,
\end{equation*}
hence $H_{0}$ cannot be rejected, the covariance matrices can be
pooled. 
\subsection*{(b)}
\label{sec:b-3}

Since we can pool the sample matrix, which is given as
\begin{equation*}
  \b S_{\rm spooled} =
  \begin{pmatrix}
    156.82 &57.13 &20.27 &22.13 &0.16 &20.40 \\ 
    57.13 &53.20 &10.93 &9.57 &-0.47 &11.67 \\ 
    20.27 &10.93 &5.95 &6.30 &-0.23 &4.67 \\ 
    22.13 &9.57 &6.30 &22.39 &-0.54 &11.37 \\ 
    0.16 &-0.47 &-0.23 &-0.54 &0.99 &0.06 \\ 
    20.40 &11.67 &4.67 &11.37 &0.06 &53.09 
  \end{pmatrix},
\end{equation*}
we can use the distance function $d_{i}(\b x_{0}) = (\b x_{0} - \mean
x_{i})^{T} \b S_{\rm spooled}^{-1} (\b x_{0} - \mean
x_{i})$ given in \cite[p. 611]{book}, where  $\b x_{0}$ is classified to
$\pi_{i}$ if $d_{i}(\b x_{0}) = \min_{i = 1, 2, 3}
\left\{
    d_{i}(\b x_{0})
\right\}$. The distance functions can be created with
\begin{equation*}
  S_{\rm pooled}^{-1} =
  \begin{pmatrix}
    0.01 &-0.01 &-0.03 &-0.00 &-0.01 &-0.00 \\ 
    -0.01 &0.04 &-0.04 &0.01 &0.01 &-0.00 \\ 
    -0.03 &-0.04 &0.42 &-0.07 &0.04 &-0.00 \\ 
    -0.00 &0.01 &-0.07 &0.07 &0.03 &-0.01 \\ 
    -0.01 &0.01 &0.04 &0.03 &1.05 &-0.01 \\ 
    -0.00 &-0.00 &-0.00 &-0.01 &-0.01 &0.02  
  \end{pmatrix},
\end{equation*}
and 
\begin{equation*}
  \mean x_{1} =
  \begin{pmatrix}
    183.10 \\ 
    129.62 \\ 
    51.24 \\ 
    146.19 \\ 
    14.10 \\ 
    104.86 
  \end{pmatrix}, \quad 
  \mean x_{2} =
  \begin{pmatrix}
    201.00 \\ 
    119.32 \\ 
    48.87 \\ 
    124.65 \\ 
    14.29 \\ 
    81.00  
  \end{pmatrix}, \quad 
  \mean x_{3} =
  \begin{pmatrix}
    138.23 \\ 
    125.09 \\ 
    51.59 \\ 
    138.27 \\ 
    10.09 \\ 
    106.59
  \end{pmatrix}.
\end{equation*}



\subsection*{(c)}
\label{sec:c-3}
Since both the means and the varaince is unkown, we use the estimates
of the errors of missclassification according to Okamato. First define 
\begin{align*}
  a_{1} &= \frac{\hat\Delta^{2} +
    12(p-1)}{16\hat\Delta}\phi(\hat\Delta) =0.99 ,  \quad a_{2} =
  \frac{\hat\Delta^{2} + 4(p-1)}{16\hat\Delta}\phi(\hat\Delta) =0.18 , \quad
  \text{and}\quad \\
  a_{3} &= \frac{\hat\Delta(p-1)}{4}\phi(\hat\Delta) = 7.70,
\end{align*}
where 
\begin{equation*}
  \hat\Delta^{2} = \frac{f - p - 1}{f} (\mean x_{1} - \mean x_{2})^{T}
  \b S_{\rm pooled}^{-1} (\mean x_{1} - \mean x_{2}) = 6.16,
  \quad f = n_{1} + n_{2} - 2 = 50.
\end{equation*}
Then the error of missclassificating a member of $\pi_{0}$ onto 
$\pi_{1}$ is given by 
\begin{equation*}
  e_{1} = P(\b x_{0} \in \pi_{2}) \approx  \phi(-\frac 12 \hat\Delta) + \frac{a_{1}}{n_{1}} +
  \frac{a_{2}}{n_{2}} + \frac{a_{3}}{f} = 0.21,  
\end{equation*}
and the error of missclassificating a member of $\pi_{0}$ onto 
$\pi_{2}$ is
\begin{equation*}
  e_{1} = P(\b x_{0} \in \pi_{1}) \approx  \phi(-\frac 12 \hat\Delta) + \frac{a_{2}}{n_{1}} +
  \frac{a_{1}}{n_{2}} + \frac{a_{3}}{f} = 0.20.
\end{equation*}

%%% Local Variables:
%%% mode: latex
%%% TeX-master: "examination"
%%% End:


\begin{comment}
  
  We could calculate the linear separators, $ l_i^T x_0 + c_i$, where
  we obtain
  \begin{align*}
    l_1^T x_0 + c_1 &= (-0.57, 1.80, 1.01, 6.33, 18.94, 0.33)x_0 + -703.95 \\ 
    l_2^T x_0 + c_2 &= (-0.14, 1.31, 1.48, 5.15, 18.31, 0.04)x_0 + -554.06 \\ 
    l_3^T x_0 + c_3 &= (-1.08, 1.89, 3.03, 5.69, 15.11, 0.51)x_0 + -618.43,
  \end{align*}
  In which we can conclude that ${\bf x_0}$ belongs to  $\pi_i$ if $
  l_i^T x_0 + c_i = \max_i \left(   l_i^T x_0 + c_i\right)
  $

\end{comment}

\section*{Exercise 4}
\label{sec:exercise-4}

\subsection*{(a)}
\label{sec:a-3}

We can check if the matrices can be pooled by using the test variable
\begin{equation*}
  \lambda^* = \frac{\Pi \det(V_i ^{f_i/2})}{\det(V^{f/2})}
  \frac{f^{pf/2}}{\Pi f_i ^{pfi/2}}
\end{equation*}
and using the Box correction, we get
\begin{equation*}
  -2  \frac{m}{f} \ln \lambda^* = 49.2750,
\end{equation*}
which we compare against the critical value
\begin{equation*}
  c = \chi^2_{0.95}(f) = 58.1240,
\end{equation*}
hence the covariance matrices can be pooled into one matrix.

\subsection*{(b)}
\label{sec:b-3}

We could calculate the linear separators, $ l_i^T x_0 + c_i$, where
we obtain
\begin{align*}
l_1^T x_0 + c_1 &= (-0.57, 1.80, 1.01, 6.33, 18.94, 0.33)x_0 + -703.95 \\ 
l_2^T x_0 + c_2 &= (-0.14, 1.31, 1.48, 5.15, 18.31, 0.04)x_0 + -554.06 \\ 
l_3^T x_0 + c_3 &= (-1.08, 1.89, 3.03, 5.69, 15.11, 0.51)x_0 + -618.43,
\end{align*}
In which we can conclude that ${\bf x_0}$ belongs to  $\pi_i$ if $
  l_i^T x_0 + c_i = \max_i \left(   l_i^T x_0 + c_i\right)
$

\subsection*{(C)}
\label{sec:c-3}

Since we have unknown mean and variance, we use the approximation 
\begin{align*}
  e_1 &= \phi(\Delta) \frac{\Delta^2 + 12(p-1)}{16\Delta} \\
  e_2 &=  \phi(\Delta)\frac{\Delta^2 - 4(p-1)}{16\Delta},
\end{align*}
where $\Delta$,  $a_1$ and $a_2$  are defined as in the lectures.
Using the pooled matrix $S_p$ for all three populations, we got the
probability of classifying wrongly into $\pi_1$ to be
\begin{equation*}
  e_1 \approx 0.0019
\end{equation*},
and for $\pi_2$ to be
\begin{equation*}
  e_2  \approx 0.0019.
\end{equation*}

%%% Local Variables:
%%% mode: latex
%%% TeX-master: "examination"
%%% End:

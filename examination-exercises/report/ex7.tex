
\subsection*{Exercise 7}
\label{sec:exercise-7}

Since there are no structure for the mean, the sample covariance matrix
is created by taking the sample covariance matrix of $\b X = [X_{1},
X_{2}, X_{3}, X_{4}]$ with dimension $(n \times p)$, where $n = 45$ and
$p= 11$, and the matrices $X_{i}$ represents the samples of the
different groups. We let $i = 1$ be the control group, $i = 2$ be
25-50r, $i = 3$ be 75-100r, and $i = 4$ be  125-250r. \\
\\
The hypothesis $H_{0}: \b\Sigma = \sigma^{2}[(1-\rho)\b I + \rho \b 1
\b 1^{T}]$ is not rejected if
\begin{equation*}
  \left[ n  - 1 - \frac{p(p+1)^2 (2p-3)}{6(p-1)(p^2+p-4)} \right] \ln
    \Lambda < \chi^{2}_{g}(0.95),
\end{equation*}
where 
\begin{equation*}
  g = \frac{1}{2}p(p+1) -2 = 64,
\end{equation*}
and
\begin{equation*}
 \ln\Lambda = \ln\abs S- \ln\frac{\b 1^{T} \b S \b 1}{p}  - (p-1)\ln\frac{p\tr S - \b 1^{T} \b S \b 1}{p(p-1)}
\end{equation*}
We get that
\begin{equation*}
  \left[ n  - 1 - \frac{p(p+1)^2 (2p-3)}{6(p-1)(p^2+p-4)} \right] \ln
    \Lambda \approx 263.05,
\end{equation*}
which is larger than
\begin{equation*}
  \chi^{2}_{g}(0.95) \approx 83.68.
\end{equation*}
So $H_{0}$ is rejected, there are no intraclass covariance matrix.

\subsection*{(b)}
\label{sec:b-6}

Further, we which to test if the correlation matrix is zero on all the
sub diagonals. We use the test found in Exercise 8.9 from \cite[p. 472]{book}. Set 
\begin{equation*}
  \ln \Lambda = \frac n2 \ln |\b S |- \frac n2 \sum_{i = 1}^{p} \ln\b (S_{ii}),
\end{equation*}
and 
\begin{equation*}
  u = 2\left(1 - \frac{2p + 11}{6n}\right).
\end{equation*}
We find that that the test size
\begin{equation*}
  -u \ln \Lambda = 1159.79  > 73.31 = c = \chi^2_{p(p-1)/2}(0.95).
\end{equation*}
Thus, it can be concluded that the scores at each time step is not
independent. 

\subsection*{(c)}
\label{sec:c-6}

We consider the growth curve model 
\begin{equation*}
  \b Y = \b{ABC} +  \b E,
\end{equation*}
where $A = (\b 1_{p},\b t)$, $\b t = (0, 1, \dots, p - 1)^{T}$, $p =
10$,  $\b B$ contains
the parametric values, where each column 
consist of the parametric values for each group,  and $\b C$ is a
matrix of controlling which group each column in $\b B$ belong to. The
matrix $\b E \sim N_{p, n}(\b 0, \b \Sigma, \b I)$. We can write $\b C$ as
\begin{equation*}
  \b C =
  \begin{pmatrix}
    \b 1_{n_{1}}^{T} & \b 0_{n_{2}}^{T} & \cdots & \b 0_{n_{k}}^{T} \\
    \b 0_{n_{1}}^{T} & \b 1_{n_{2}}^{T} & \cdots & \b 0_{n_{k}}^{T} \\
    \b 0_{n_{1}}^{T} & \b 0_{n_{2}}^{T} & \cdots & \b 1_{n_{k}}^{T}
  \end{pmatrix},
\end{equation*}
where $k = 4$ is the number of groups. \\
\\
The solution of finding the parameters in $\b B$ is given by 
\begin{align*}
  \b \hat B &= (\b A^{T}\b V^{-1}\b A)^{-1}  \b A^{T}\b V^{-1}\b Y
              \b C^{T}(\b C\b C^{T})^{-1} % looks great!
  \\
  &=
  \begin{pmatrix}
    117.39 &96.64 &121.25 &150.95 \\ 
    7.38 &5.43 &5.11 &5.34
  \end{pmatrix},
\end{align*}
where
\begin{equation*}
  \b V =  \b Y  (\b I - \b P_{c})  \b Y^{T},
\end{equation*}
where
\begin{equation*}
  \b P_{c} = \b C^{T}(\b C\b C^{T})^{-1}\b C.
\end{equation*}
% The oter matrices are to large to be displayed here. 

\subsection*{(d)}
\label{sec:d-2}

To compare the parameters we consider three different tests,
\begin{equation*}
  H_i: \b B^{(1)} - \b B^{(i)} = \b 0,\quad i = 2,3,4,
\end{equation*}
where $\b B^{(i)}$ denotes column $i$ of $\b B$. We can reformulate the
hypothesis $i$ as
\begin{equation*}
  H_i: \b{G B H_i} = \b 0, \quad i = 2,3,4,
\end{equation*}
where the matrix $\b H_i$ has its first element equal to 1 and its $i$th element
equal to $-1$. The matrix $\b G$ is just the identity matrix of
dimension 2. We reject $H_{i}$ if 
\begin{equation*}
  \Lambda_{i} = \frac{\abs{\b G (\b A^{T} \b V^{-1} \b A)^{-1}\b G^{T}}}
  {\abs{\b G (\b A^{T} \b V^{-1} \b A)^{-1} \b G ^{T} + \b G \b \hat B\b H_{i}(\b
      H_{i}^{T} \b R \b H_{i})^{-1}\b H_{i}^{T} \b \hat B^{T} \b G^{T}}}
\end{equation*}
where
\begin{align*}
  \b R = (\b C \b C^{T})^{-1} + ( &\b C \b C^{T})^{-1} CY^{T} \\
 &  \cdot\left[
    \b V^{-1}  - \b V^{-1}\b A(\b A^{T} \b V^{-1} \b A)^{-1}\b A^{T} \b V^{-1}
  \right] 
  \b Y \b C^{T}(\b C \b C^{T})^{-1}.
\end{align*}
Then we get the  asymptotic distribution given by
\begin{equation*}
 Q_{i} = - \left(n - k + q - p - \frac{1}{2}(r - t + 1)\right) \ln \Lambda_{i} \sim \chi^{2}_{rt}, 
\end{equation*}
where $t=1$ is the number of columns in $\b H_{i}$, $r = 4$ is the
number of rows in $\b H_{i}$, for all $i = 2,3,
4$, and $q = 2$ is the number of columns in $\b G$. We get that
\begin{center}
  \begin{tabular}{r|c}
    $i$   & $Q_{i}$   \\ \hline
    2   & 1.24 \\
    3   & 1.22 \\
    4   & 2.15 \\
  \end{tabular}
\end{center}
$H_{i}$ is rejected  if $Q_{i}$ is larger than
$\chi^{2}_{rt}(0.95) \approx 6.00$, for all $i = 2,3,4$. Thus we do not reject any hypothesis. 
%%% Local Variables:
%%% mode: latex
%%% TeX-master: "examination"
%%% End:

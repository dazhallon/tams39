
\section*{Exercise 6}
\label{sec:exercise-6}

\subsection*{(a)}
\label{sec:a-5}


The foll wing linear model was designed
\begin{equation*}
  \b Y = \b{XB} + \b E,
\end{equation*}
where $\b Y$ is the out parameters, $\b X = (\b 1_n, \b x_1, \b x_2)$
is the design matrix, and $\b B:(3 \times 4) $ contained the unknown parameter values, which
was estimated using linear regression as
\begin{equation*}
  \b\hat B =
  \begin{pmatrix}
    -94.60 &-228.67 &-258.80 &-246.05 &-232.96 \\ 
    19.27 &43.65 &48.99 &47.33 &45.10 \\ 
    -24.68 &-20.59 &3.32 &5.22 &8.24 
  \end{pmatrix}.
\end{equation*}

\subsection*{(b)}
The sample correlation matrix was 
\begin{equation*}
  \b R =
  \begin{pmatrix}
    1.00 &0.86 &0.71 &0.64 &0.65 &0.81 &-0.09 \\ 
    0.86 &1.00 &0.91 &0.89 &0.90 &0.95 &0.10 \\ 
    0.71 &0.91 &1.00 &0.98 &0.98 &0.92 &0.22 \\ 
    0.64 &0.89 &0.98 &1.00 &1.00 &0.90 &0.22 \\ 
    0.65 &0.90 &0.98 &1.00 &1.00 &0.90 &0.24 \\ 
    0.81 &0.95 &0.92 &0.90 &0.90 &1.00 &0.22 \\ 
    -0.09 &0.10 &0.22 &0.22 &0.24 &0.22 &1.00  
  \end{pmatrix}.
\end{equation*}
was is interesting is that the time steps that are close to each other
correlates more than those times further apart. We can even note that
the first and last time steps have negative correlation an is almost 0.

\subsection*{(c)}
\label{sec:c-5}

The average weight is governed by the last row in $\b B$, $\b B_{2}$, so therefore we
want to test $H_{0}: \b B_{2} = \b 0$. According to Result 7.11 in
\cite[p. 396]{book}, the  following modified statistic is rejected if it
is large:

\begin{equation*}
  -\left[n - r- 1 - \frac{1}{2}(m-r+q+1)\right]\ln
  \left(
    \frac{\abs{\b\hat\Sigma}}{\abs{\b\hat\Sigma_{1}}}
  \right) \sim \chi^{2}_{m(r-q)},
\end{equation*}
where $n= 25$ is the sample size, $r=2$ is such that $\rank {X} = r + 1$,
$m=4$ is the number of outputs, and $q=1$ is the number of rows in
$B_{2}$. The matrices $\b \hat \Sigma$ is the standard MLE, and
$\b\hat\Sigma_{1}$ is the MLE under $H_{0}$. We get $\b\hat\Sigma_{1}$
by solving the least squared problem that $H_{0}$ yield. So we get that
\begin{equation*}
  \b\hat B_{1} = (\b X_{1}^{T} \b X_{1})^{-1}\b X_{1} \b Y, \quad
  \text{and} \quad 
  \b\hat\Sigma_{1} = \frac{1}{n} (\b Y  - \b X_{1} \b\hat B_{1})^{T}(\b
  Y  - \b X_{1} \b \hat B_{1}).
\end{equation*}
Then
\begin{equation*}
    -\left[n - r- 1 - \frac{1}{2}(m-r+q+1)\right]\ln
  \left(
    \frac{\abs{\b\hat\Sigma}}{\abs{\b\hat\Sigma_{1}}}
  \right) =8.31, \quad \text{and}\quad \chi^{2}_{m(r-q)}(1 - \alpha) = 11.07,
\end{equation*}
hence $H_{0}$ cannot be rejected. 

\begin{comment}
To test what effect the average weight to fish has on the the regression
model we consider $H_{0}: \b C \b B = \b B_{2} = \b 0$. Where the
matrix $\b C$ is a matrix of dimension $(r - q)\times (r+1)$, where
$\rank X = r + 1$, and $q$ is the number of 

We can the test if the \textit{average weight} affect the causation of
death by considering the test where each element of the bottom row of $\b
B$ should be equal to zero. We get the hypothesis
\begin{equation*}
  H_0:\ \b{CB} = \b 0,
\end{equation*}
where 
\begin{equation*}
  C = 
  \begin{pmatrix}
    0 &0 &0 \\ 
    0 &0 &0 \\ 
    0 &0 &0 \\ 
    0 &0 &0 \\ 
    0 &0 &1   
  \end{pmatrix},
\end{equation*}
form which could use the test described on page 11 of Lecture 9, and use
the corresponding asymptotic expansion found on page 12 of Lecture 9 to
test on a significance level of 5 \%. We find that 
\begin{align*}
  p &= P(\chi^2 \geq z) + (\gamma/\nu^2)P(\chi^2 \geq z) -  P(\chi^2
      \geq z )\\
    &\approx 0.7931 > 0.05,
\end{align*}
hence we reject $H_0$.
\end{comment}
%%% Local Variables:
%%% mode: latex
%%% TeX-master: "examination"
%%% End:

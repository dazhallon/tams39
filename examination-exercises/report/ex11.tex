
\section*{Exercise 11}
\label{sec:exericse-11}

\subsection*{(a)}
\label{sec:a-7}

The sample correlation matrix is
\begin{equation*}
  \b R =
  \begin{pmatrix}
    1.00 &0.87 &-0.37 &-0.39 &-0.49 &-0.23 \\ 
    0.87 &1.00 &-0.35 &-0.55 &-0.65 &-0.19 \\ 
    -0.37 &-0.35 &1.00 &0.15 &0.23 &0.03 \\ 
    -0.39 &-0.55 &0.15 &1.00 &0.70 &0.50 \\ 
    -0.49 &-0.65 &0.23 &0.70 &1.00 &0.67 \\ 
    -0.23 &-0.19 &0.03 &0.50 &0.67 &1.00  
  \end{pmatrix}.
\end{equation*}
The weight and waist size correlates strongly, and the pulse seem to
decrease the smaller the person is. \\
\\
What is interesting is that the exercises correlates strongly, but
perhaps not as much as we might first think. \\
\\
We can also note that traits for a heavier person (large waist and high
weight) affect negatively on all the exercises, which is most likely
because they are body exercises and do not use any weights.

\subsection*{(b)}
\label{sec:b-10}

The sample canonical correlations are $\b \alpha_1 = (0.44, -0.90, 0.03
)^T$, $\b \beta_1~=~(-0.26, -0.80, 0.54 )^T$, and $\b \alpha = (-0.16 ,0.43 ,0.89  
)^T$ and $\b \beta_2 = (-0.70 ,0.67 ,-0.23 )^T$. \\
\\
These are harder to interpret. From \cite[pp.545-547]{book} it is
suggested that we compute the covariance between $\alpha_1$ and $X_1$,
etc.. Then we find that 
\begin{align*}
  \cov {\b X_1}{\b \alpha_1} &=
  \begin{pmatrix}
    -31.00 &-5.99 &4.86 
  \end{pmatrix} \\
  \cov {\b X_2}{\b \beta_1} &=
  \begin{pmatrix}
    -55.23 &-734.65 &-119.42  
  \end{pmatrix}
\end{align*}
Here we can see that that the pulse and waist size has quite low effect
on the variation  whereas the other variables as a stronger
correlation. It is interesting that most of these covariance are
negative.  

\subsection*{(c)}
\label{sec:c-9}

We test the hypothesis $H_0: \rho_{k+1} = \dots = \rho_{p} = 0,$ for $k
= 0,1,2$, by using the test suggested in Lecture 11. Let 
\begin{equation*}
  \b A =\b S_{11}^{-\frac 12} \b S_{12}\b S_{22}^{-\frac 12},
\end{equation*}
set the LRT as
\begin{equation*}
  \ln \Lambda = \sum_{i = k+1}^{p} \ln (1 - r^2_{i}),
\end{equation*}
and use the correction 
\begin{equation*}
  u =  n - k \frac 12(p + q  + 3) - \sum_{i = 1}^{k} r^2_i,
\end{equation*}
suggested by Fujikoshi, we get \\
\\
\begin{center}
\begin{tabular}[h]{c|cc|c}
 $k$ &  $Q$ & $c$ & Conclusion\\ \hline
  0& 20.97 &16.92 & Reject\\ 
 1 & 16.17 &9.49 & Reject\\ 
 2 & 10.98 &3.84 & Reject
\end{tabular}
\end{center}
where 
\begin{equation*}
  Q = -u \ln \Lambda,
\end{equation*}
and
\begin{equation*}
  c = \chi^2_{0.95}((p-k)(q-k)).
\end{equation*}

\subsection*{Redoing the exercise but with normalized samples}
\label{sec:redoing-exercise-but}

The normalized samples, $\b Z$, of $\b X$ are given by
\begin{equation*}
  \b Z_i = \frac{\b X_i - \bar {\b X}_i}{\b S_{ii}}, \quad i = 1, \dots, p = 6.
\end{equation*}
We get that
\begin{equation*}
  \b Z =
  \begin{pmatrix}
    0.50 &0.19 &-0.85 &-0.84 &0.26 &-0.20 \\ 
    0.42 &0.50 &-0.57 &-1.41 &-0.57 &-0.20 \\ 
    0.58 &0.81 &0.26 &0.48 &-0.71 &0.60 \\ 
    -0.67 &-0.12 &0.82 &0.48 &-0.65 &-0.65 \\ 
    0.42 &-0.12 &-1.40 &0.67 &0.15 &-0.24 \\ 
    0.14 &0.19 &-0.01 &-1.03 &-0.71 &-0.55 \\ 
    1.31 &0.81 &-0.01 &-0.27 &-0.71 &-0.63 \\ 
    -0.47 &-0.44 &0.54 &-0.65 &-0.33 &-0.59 \\ 
    -0.11 &-1.37 &2.48 &1.05 &0.87 &-0.59 \\ 
    -1.00 &-0.75 &-0.01 &1.43 &1.69 &3.50 \\ 
    -0.39 &-0.44 &-0.85 &1.43 &-0.41 &-0.63 \\ 
    -0.51 &-0.75 &-0.57 &0.67 &1.03 &0.87 \\ 
    -1.00 &-0.44 &1.10 &0.86 &1.11 &0.68 \\ 
    2.77 &3.31 &-0.85 &-1.60 &-1.53 &-0.40 \\ 
    0.58 &0.19 &-1.40 &-0.65 &-1.21 &-0.77 \\ 
    0.95 &0.50 &0.82 &0.48 &1.03 &0.97 \\ 
    -0.11 &0.50 &-0.29 &-1.03 &-1.37 &-0.88 \\ 
    -0.87 &-1.06 &-0.57 &0.29 &1.35 &0.19 \\ 
    -0.92 &-0.75 &-0.29 &1.05 &1.27 &0.05 \\ 
    -1.64 &-0.75 &1.65 &-1.41 &-0.57 &-0.53  
  \end{pmatrix}
\end{equation*}
The sample correlation matrix is now
\begin{equation*}
  \b R =
  \begin{pmatrix}
    1.00 &0.87 &-0.37 &-0.39 &-0.49 &-0.23 \\ 
    0.87 &1.00 &-0.35 &-0.55 &-0.65 &-0.19 \\ 
    -0.37 &-0.35 &1.00 &0.15 &0.23 &0.03 \\ 
    -0.39 &-0.55 &0.15 &1.00 &0.70 &0.50 \\ 
    -0.49 &-0.65 &0.23 &0.70 &1.00 &0.67 \\ 
    -0.23 &-0.19 &0.03 &0.50 &0.67 &1.00
  \end{pmatrix}.
\end{equation*}
which is the same matrix as for $\b X$.\\
\\
The canonical correlation variables is now

\begin{align*}
  \b \alpha_{1} &=
                  \begin{pmatrix}
                    0.44 &-0.90 &0.03  
                  \end{pmatrix}^{T} \\
  \b \beta_{1} &=
                 \begin{pmatrix}
                   -0.26 &-0.80 &0.54  
                 \end{pmatrix}^{T} \\
  \b \alpha_{2} &=
                  \begin{pmatrix}
                    -0.16 &0.43 &0.89  
                  \end{pmatrix}^{T} \\
  \b \beta_{2} &=
                 \begin{pmatrix}
                   -0.70 &0.67 &-0.23 
                 \end{pmatrix}^{T}
\end{align*}
Now we can see clearer how the canonical correlations relates to the
data. The first canonical pair may tell us that the pulse how the
middle-aged man might not cause much difference in how well the
exercises are performed. 
 In the first canonical pair, we have the correlations
\begin{align*}
  \corr {\b X_{1}}{ \b u_{1}} &=
  \begin{pmatrix}
    -0.35 &-0.53 &0.19 
  \end{pmatrix}\\
  \corr {\b X_{2}}{ \b v_{1}} &=
  \begin{pmatrix}
    -0.55 &-0.62 &-0.12 
  \end{pmatrix}
\end{align*}
\\
Lastly, we do the same test as before. This time we had the following
results:
\begin{center}
\begin{tabular}[h]{c|cc|c}
  $k$ & $Q$ & $c$ & Conclusion \\ \hline
0.00 &20.05 &16.92 & Reject \\ 
1.00 &15.53 &9.49 & Reject\\ 
2.00 &10.98 &3.84  & Reject
\end{tabular}
\end{center}
That is we can not conclude that any correlation $\rho$ is zeros.
%%% Local Variables:
%%% mode: latex
%%% TeX-master: "examination"
%%% End:

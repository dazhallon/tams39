
\section*{Exercise 11}
\label{sec:exericse-11}

\subsection*{(a)}
\label{sec:a-7}

The sample correlation matrix is
\begin{equation*}
  \b R =
  \begin{pmatrix}
    1.00 &0.87 &-0.37 &-0.39 &-0.49 &-0.23 \\ 
    0.87 &1.00 &-0.35 &-0.55 &-0.65 &-0.19 \\ 
    -0.37 &-0.35 &1.00 &0.15 &0.23 &0.03 \\ 
    -0.39 &-0.55 &0.15 &1.00 &0.70 &0.50 \\ 
    -0.49 &-0.65 &0.23 &0.70 &1.00 &0.67 \\ 
    -0.23 &-0.19 &0.03 &0.50 &0.67 &1.00  
  \end{pmatrix}.
\end{equation*}
The weight and waist size correlates strongly, and the pulse seem to
decrease the smaller the person is. \\
\\
What is interesting is that the exercises correlates strongly, but
perhaps not as much as we might first think. \\
\\
We can also note that traits for a heavier person (large waist and high
weight) affect negatively on all the exercises, which is most likely
because they are body exercises and do not use any weights.

\subsection*{(b)}
\label{sec:b-10}

Using the same results as in Exercise 10(a), we get that the sample
canonical correaltions are
\begin{equation*}
  \rho_{1} =0.7956 , \quad \rho_{2} =  0.2006,\quad  \rho_{3} = 0.0726,
\end{equation*}
where the sample cancocial coefficients where
\begin{equation*}
  \b \alpha_{1} =
  \begin{pmatrix}
    -0.03 \\ 
    0.49 \\ 
    -0.01
  \end{pmatrix}, \quad
  \b \alpha_{2} =
  \begin{pmatrix}
    0.08 \\ 
    -0.37 \\ 
    0.03      
  \end{pmatrix}, \quad
  \b \alpha_{3} =
  \begin{pmatrix}
    -0.01 \\ 
    0.16 \\ 
    0.15
  \end{pmatrix}
\end{equation*},
and

\begin{equation*}
  \b \beta_{1} =
  \begin{pmatrix}
    0.07 \\ 
    0.02 \\ 
    -0.01 
  \end{pmatrix}, \quad
  \b \beta_{2} =
  \begin{pmatrix}
    0.07 \\ 
    0.00 \\ 
    0.02  
  \end{pmatrix}, \quad
  \b \beta_{3} =
  \begin{pmatrix}
    0.25 \\ 
    -0.02 \\ 
    0.01  
  \end{pmatrix}.
\end{equation*}

\subsection*{(c)}
\label{sec:c-9}

We test the hypothesis $H_0: \rho_{k+1} = \dots = \rho_{p} = 0,$ for $k
= 0,1,2$, $p = 3$, by using the test found in
\cite[p. 565]{book}. Consider the hypothesises $H_{0}^{(k)}:
\rho_{k+1},\dots, \rho_{p} = 0$, for $k = 0,1,2$ and $p =
3$. $H_{0}^{(k)}$ is rejected if 
\begin{equation}\label{eq:test_ex11}
  -\left(n - 1  - \frac{1}{2}(p+q+1)\right) \sum_{i = k+1}^{p} \ln (1 -
  \rho_{i}^{2}) > \chi^{2}_{(p-k)(1-k)}(1-\alpha), \quad \alpha = 0.05.
\end{equation}
Deonte $Q$ and $c$ as the LHS and the RHS of \eqref{eq:test_ex11}
respectivley. We present the results in Table \ref{tab:test_ex11}. 
\begin{table}
  \centering
  \begin{tabular}{l|ccc}
    $k$&$Q$&$c$& Reject $H_{0}^{k}$ \\ \hline
    0 &19.93 &16.92 & yes \\ 
    1 &0.72 &9.49 & no \\ 
    2 &0.06 &3.84 & no  
  \end{tabular}
  \caption{Results from the tests}
  \label{tab:test_ex11}
\end{table}
We make the conclussion that the only non-zero sample canonical
correlation is $\rho_{1}$.
\subsection*{Redoing the exercise but with normalized samples}
\label{sec:redoing-exercise-but}

The normalized samples, $\b Z$, of $\b X$ are given by
\begin{equation*}
  \b Z_i = \frac{\b X_i - \bar {\b X}_i}{\b S_{ii}}, \quad i = 1, \dots, p = 6.
\end{equation*}
We get that
\begin{equation*}
  \b Z =
  \begin{pmatrix}
    0.50 &0.19 &-0.85 &-0.84 &0.26 &-0.20 \\ 
    0.42 &0.50 &-0.57 &-1.41 &-0.57 &-0.20 \\ 
    0.58 &0.81 &0.26 &0.48 &-0.71 &0.60 \\ 
    -0.67 &-0.12 &0.82 &0.48 &-0.65 &-0.65 \\ 
    0.42 &-0.12 &-1.40 &0.67 &0.15 &-0.24 \\ 
    0.14 &0.19 &-0.01 &-1.03 &-0.71 &-0.55 \\ 
    1.31 &0.81 &-0.01 &-0.27 &-0.71 &-0.63 \\ 
    -0.47 &-0.44 &0.54 &-0.65 &-0.33 &-0.59 \\ 
    -0.11 &-1.37 &2.48 &1.05 &0.87 &-0.59 \\ 
    -1.00 &-0.75 &-0.01 &1.43 &1.69 &3.50 \\ 
    -0.39 &-0.44 &-0.85 &1.43 &-0.41 &-0.63 \\ 
    -0.51 &-0.75 &-0.57 &0.67 &1.03 &0.87 \\ 
    -1.00 &-0.44 &1.10 &0.86 &1.11 &0.68 \\ 
    2.77 &3.31 &-0.85 &-1.60 &-1.53 &-0.40 \\ 
    0.58 &0.19 &-1.40 &-0.65 &-1.21 &-0.77 \\ 
    0.95 &0.50 &0.82 &0.48 &1.03 &0.97 \\ 
    -0.11 &0.50 &-0.29 &-1.03 &-1.37 &-0.88 \\ 
    -0.87 &-1.06 &-0.57 &0.29 &1.35 &0.19 \\ 
    -0.92 &-0.75 &-0.29 &1.05 &1.27 &0.05 \\ 
    -1.64 &-0.75 &1.65 &-1.41 &-0.57 &-0.53  
  \end{pmatrix}
\end{equation*}
The sample correlation matrix is now
\begin{equation*}
  \b R =
  \begin{pmatrix}
    1.00 &0.87 &-0.37 &-0.39 &-0.49 &-0.23 \\ 
    0.87 &1.00 &-0.35 &-0.55 &-0.65 &-0.19 \\ 
    -0.37 &-0.35 &1.00 &0.15 &0.23 &0.03 \\ 
    -0.39 &-0.55 &0.15 &1.00 &0.70 &0.50 \\ 
    -0.49 &-0.65 &0.23 &0.70 &1.00 &0.67 \\ 
    -0.23 &-0.19 &0.03 &0.50 &0.67 &1.00
  \end{pmatrix}.
\end{equation*}
which is the same matrix as for $\b X$.\\
\\
The sample canonical correlations where 
\begin{equation*}
  \rho_{1} = 0.7956,\quad  \rho_{2} =   0.2006,\quad \rho_{3} =  0.0726
\end{equation*},
with canonical coefficients
\begin{equation*}
  \b \alpha_{1} =
  \begin{pmatrix}
    -0.78 \\ 
    1.58 \\ 
    -0.06 
  \end{pmatrix}, \quad
  \b \alpha_{2} =
  \begin{pmatrix}
    -0.03 \\ 
    0.49 \\ 
    -0.01
  \end{pmatrix}, \quad
  \b \alpha_{3} =
  \begin{pmatrix}
    -0.19 \\ 
    0.51 \\ 
    1.05  
  \end{pmatrix},
\end{equation*}
and

\begin{equation*}
  \b \beta_{1} =
  \begin{pmatrix}
    0.35 \\ 
    1.05 \\ 
    -0.72  
  \end{pmatrix}, \quad
  \b \beta_{2} =
  \begin{pmatrix}
    -0.38 \\ 
    0.12 \\ 
    1.06  
  \end{pmatrix}, \quad
  \b \beta_{3} =
  \begin{pmatrix}
    1.30 \\ 
    -1.24 \\ 
    0.42  
  \end{pmatrix}
\end{equation*}
Finally we use test the same hypothesises described in Exercise
(c). The results are presented in Table
\ref{tab:test_ex11_standardized}. So we can conclude that only
$\rho_{1}$ is non-zero of all the sample canonical correlations. 
\begin{table}
  \centering
  \begin{tabular}{l|ccc}
    $k$&$Q$&$c$& Reject $H_{0}^{k}$ \\ \hline
    0 &19.93 &16.92 & yes \\ 
    1 &0.72 &9.49 & no \\ 
    2 &0.06 &3.84 & no  
  \end{tabular}
  \caption{Results from the tests}
  \label{tab:test_ex11_standardized}
\end{table} 
%%% Local Variables:
%%% mode: latex
%%% TeX-master: "examination"
%%% End:

\subsection*{Exercise 9}
\label{sec:exercise-9}

We consider the the following model for $k$ factors
\begin{equation*}
  \b X = \b \mu  + \b{LF} + \b \epsilon,
\end{equation*}
where $\b L: (p\times m)$ are the loading factors, which are
deterministic. $\b F = (F_{1}, \dots, F_{m})$ and $\b \epsilon =
(\epsilon_{1}, \dots, \epsilon_{p})$ are $m + p$ unobserved
values which we assume has the structure
\begin{align*}
  E(\b F) &= \b 0,\quad \cov{\b F}{\b F} = \b I \\
  E(\b \epsilon) &= \b 0,\quad \cov{\b \epsilon}{\b \epsilon}  = \b \Psi =
  \diag(\psi_{1}, \dots, \psi_{p})
\\
  \cov{\b \epsilon}{\b F} &= \b 0.
\end{align*}
finally, we assume that $\b \epsilon \sim N(\b 0, \b \Psi)$. So that
$\b X \sim N(\b \mu, \b \Sigma)$, where $\b \Sigma = \b L \b L ^{T} +
\b \Psi$. 
\subsection*{(b)}
\label{sec:b-8}

We note that a PCA of the correlation matrix yields that the cumulative
coverage of the principal values are 
\begin{equation*}
  \begin{pmatrix}
    31.00 &41.04 &49.73 &56.15 &61.97 &67.65 &72.84 &77.67 &81.76\\
    85.68 &88.96 &91.96 &94.86 &97.65 &100.00 
  \end{pmatrix}\quad (\%)
\end{equation*}
so we can expected that we need quite a few factors to cover the total
variance.  An initial guess, is that we want to cover at least 85 \% of
the total variance. This gives a vague indication that $m = 10$
factors can be used. On the other hand, according to
\cite[p. 485]{book}, the variance of $\b X$ as $p(p+1)/2$ different
covariances, which we approximate by
\begin{equation*}
  \b \Sigma = \b L \b L^{T} + \Psi,
\end{equation*}
meaning that $p(p+1)/2$ covariance should be reproduced by $pm$ factor
loadnings from $\b L \b L^{T}$, plus $p$ variances from $\b
\Psi$. Considering the equation 
\begin{equation*}
  p(p + 1)/2 = pm + p,
\end{equation*}
gives $m =(p + 1)/2 - 1 = 7 $ factors, since $p = 15$. Finally, we can
also note that the eigenvalues given to us in the PCA is
\begin{equation*}
  \b \lambda =
  \begin{pmatrix}
    0.35 &0.42 &0.43 &0.45 &0.49 &0.59 &0.61 &0.72 &0.78 &0.85 \\ 0.87 &0.96 &1.30 &1.51 &4.65 
  \end{pmatrix}
\end{equation*}
where only the 3 largest eigenvalues are greater or equal to one. This
can be seen as an indication that only 3 factors needs to be used. \\
\\
As a conclusion, we recommend to use 3 factors to explain the data.
\subsection*{(c)}
\label{sec:c-8}
 We set up the test which is
according to \cite[p. 502]{book}
\begin{align*}
  \ln \Lambda &= \ln(\det(\b R)) - \ln(\det(\b \Sigma)) \\
  u &=  n - 1 -(2p+4m+5)/6 \\
  g &= \frac 12 ((p-m)^2 - (p+m)) \\
  Q &= -u \ln \Lambda \\
  c &=  \chi^2_{0.95}(g),
\end{align*}
where $\b \Sigma = \b{LL^T} + \b \Psi$. We get rejection at
$m = 2$ since $c = 97.35$ and $Q = 132.37$, but we find that $m = 3$
is adequate, since $c = 82.53$ and $Q = 71.42$. 
%%% Local Variables:
%%% mode: latex
%%% TeX-master: "examination"
%%% End:

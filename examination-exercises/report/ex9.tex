
\subsection*{Exercise 9}
\label{sec:exercise-9}

We consider the the following model for $k$ factors
\begin{equation*}
  \b X = \b \mu  + \b{LF} + \b \epsilon,
\end{equation*}
where $\b L: (p\times k)$ are the unknown factors.

\subsection*{(b)}
\label{sec:b-8}

According to the upper limit given in Lecture 10, we should let $k$ be
at most equal 10.

\subsection*{(c)}
\label{sec:c-8}

$k = 10$ is to high as we shall see. We set up the test
\begin{align*}
  \ln \Lambda &= \ln(\det(\b R)) - \ln(\det(\b \Sigma)) \\
  u &=  n - (2p+4k+11)/6 \\
  g &= \frac 12 ((p-k)^2 - (p+k)) \\
  Q &= -u \ln \Lambda =   0.0539\\
  c &= 0 = \chi^2_{0.95}(g),
\end{align*}
where $\b \Sigma = \b{LL^T} + \b \Psi$. Since $Q > c$ we reject the
hypothesis. However, if we were to choose $k=9$, we get $Q = 0.7723$
and $c =  12.5916$, thus we conclude that we can choose $k = 9$.
%%% Local Variables
%%% mode: latex
%%% TeX-master: "examination"
%%% End:

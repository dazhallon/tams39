
\section*{Exercise 10}
\label{sec:exercise-10}

\subsection*{(a)}
\label{sec:a-6}

The sample canonical correlations is $\b\alpha_1  = (0.93, -0.37  )^T$,
$\b \beta_2 = (-0.54, -0.84 )^T$, $\b \alpha_2 = (-0.28,  0.96 )^T$ and $\b
\beta_2 = (-0.91,0.40)^T$. where $\b \alpha_1$ and $\b \alpha_2$ are the
ordered eigenvectors of 
\begin{equation*}
  \b{\Sigma^{-1}_{11}\Sigma_{12}\Sigma_{22}^{-1}\Sigma_{21}}, 
\end{equation*}
and $\b \beta_2$ and $\b \beta_2$ are  the ordered eigenvectors of 
\begin{equation*}
  \b{\Sigma^{-1}_{22}\Sigma_{21}\Sigma_{11}^{-1}\Sigma_{12}}.
\end{equation*}

\subsection*{(b)}
\label{sec:b-9}

The first canonical variables are given by
\begin{align*}
  u_1 &= \b \alpha_1^T \b x_1= 0.93 x_1^{(1)} - 0.37 x_2^{(1)} \\
  v_1 &= \b \beta_1^T \b x_2 = -0.54 x_1^{(2)} - 0.84 x_2^{(2)}.
\end{align*}
We can first notice how the only variable that causes any of the
canonical variables are $x_1^{(1)}$ which represents \textit{1973
  nonprimary homicides}. From these we draw the conclusion that the
number of homicides varies the most among the different states, while
for the other crimes, we can imagine that the as the other variables
gets larger, there is less difference among the states. Thus we can
expected these types of crimes being more evenly distributed among the states.

%%% Local Variables:
%%% mode: latex
%%% TeX-master: "examination"
%%% End:

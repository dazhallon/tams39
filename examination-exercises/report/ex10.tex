
\section*{Exercise 10}
\label{sec:exercise-10}

\subsection*{(a)}
\label{sec:a-6}

According to Result 10.1 in \cite[p. 541]{book}, the sample canonical coefficients is given by finding the eigenpairs of
\begin{equation*}
  \b R_{11}^{-1/2} \b R_{12} \b R_{22}^{-1/2} \b R_{21} \b R_{11}^{-1/2}
\end{equation*}
which we done with $(\rho^{2}_{i}, \b e_{i})$, $i =1,2$, which gives us the
coefficients for $\b X^{(1)}$ by $\b\alpha_{i} = \b R_{11}^{-1/2} \b e_{i}$, $i =1,2$. Similarly the eigenpairs of
\begin{equation*}
  \b R_{22}^{-1/2} \b R_{21} \b R_{11}^{-1/2} \b R_{12} \b R_{22}^{-1/2},
\end{equation*}
denoted by $(\rho^{2}_{i}, \b f_{i})$, $i =1,2$, are the canonical coefficients for $\b
X^{(2)}$ is given by $\b\beta_{i} = \b R_{22}^{-1/2} \b f_{i}$, $i= 1,2$. It
follows that the canonical correlations are given by
\begin{align*}
  \rho_{1} &=  \sqrt{0.1248} =  0.3533\\
  \rho_{2} &=  \sqrt{0.0003} = 0.0158.
\end{align*}

\subsection*{(b)}
\label{sec:b-9}

The canonical coefficients  was calculated to be
\begin{align*}
  \begin{matrix}
     \b \alpha_{1} =  (1.22,   -0.48)^{T},  & \b \beta_{1} =  (0.62,0.97)^{T} \\
  \b \alpha_{2} =  (-0.34, 1.17)^{T},  & \b \beta_{2} = (-0.83, 0.37)^{T}
  \end{matrix}
\end{align*}, 
and so the first canonical pair is
\begin{equation*}
  \hat u_1 = 1.22 x_{1}^{(1)} - 0.48 x_{2}^{(1)} , \quad \hat v_{1} =
  0.62 x_{1}^{(2)} + 0.97 x_{2}^{(2)}. 
\end{equation*}
We can see that $\hat u_{1}$ represents mostly the number of homicides
 (1973) while $\hat v_{1}$ represented mostly the certainty for
 punishment (1970). We can investigate further by computing the sample
 correlation between $\hat u_{1}$ and $x^{(1)}$, as well as the the
 sample covariance between $\hat v_{1}$ and $x^{(1)}$, $\hat u_{1}$ and
 $x^{(2)}$, and also $\hat v_{1}$ and  $x^{(2)}$ (see \cite[p. 552]{book}). We get that
 \begin{align*}
   R_{\hat u_{1}, x^{(1)}} &= \begin{pmatrix}0.92 &0.27   \end{pmatrix} \\
   R_{\hat v_{1}, x^{(2)}} &= \begin{pmatrix}-0.93 &0.60   \end{pmatrix} \\
   R_{\hat u_{1}, x^{(2)}} &= \begin{pmatrix}-0.13 &-0.28 \end{pmatrix} \\   
   R_{\hat v_{1}, x^{(1)}} &= \begin{pmatrix}-0.01 &-0.02   \end{pmatrix} 
 \end{align*}
This tells us also that nonprimary homicides also correlates strongly
with, as well as $\hat v_{1}$ also correlates very strongly with the
certainty of punishment.\\
\\
The final interpretation of $\hat u_{1}$ is that it represents the
non-primary homicides, and $\hat v_{1}$ is concluded as the certainty
for punishment, as both $\hat u_{1}$ correlates not that much with the
primal homicides, and $\hat v_{1}$ do not correlate strongly with the
severity of the punishment. . 



\begin{comment}
  The first canonical variables are given by
  \begin{align*}
    u_1 &= \b \alpha_1^T \b x_1= 0.93 x_1^{(1)} - 0.37 x_2^{(1)} \\
    v_1 &= \b \beta_1^T \b x_2 = -0.54 x_1^{(2)} - 0.84 x_2^{(2)}.
  \end{align*}
  We can first notice how the only variable that causes any of the
  canonical variables are $x_1^{(1)}$ which represents \textit{1973
    nonprimary homicides}. From these we draw the conclusion that the
  number of homicides varies the most among the different states, while
  for the other crimes, we can imagine that the as the other variables
  gets larger, there is less difference among the states. Thus we can
  expected these types of crimes being more evenly distributed among the states.
\end{comment}
%%% Local Variables:
%%% mode: latex
%%% TeX-master: "examination"
%%% End:
